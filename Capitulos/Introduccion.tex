\chapter{Introducción}\label{cap:Introduccion}

Se suele decir que el cerebro es la red mas compleja que conoce el humano. Un cerebro humano esta compuesto por 100 billones ($10^{11}$) neuronas conectadas por 100 trillones ($10^{14}$) sinapsis, que están organizadas en múltiples escalas espaciales y son funcionalmente interactivas en múltiples escalas temporales. Las neuronas por si mismas no pueden llevar a cabo las complejas funciones de un cerebro, pero es cuando estas se conectan y organizan un sistema nervioso cuando los pensamientos, la memoria, la conciencia y los sentimientos se vuelven realidad. Muchas patologías neuronales están asociadas a daños en la red cerebral. La ubicación, extensión y caracterización de esta alteración puede permitir predecir la gravedad de la patología así como mejorar las probabilidades de recuperación del paciente. Teniendo en cuenta esto, no es sorpresa que entender la conectividad de la red cerebral y como surgen las diversas funciones cerebrales a partir de esta sea uno de los principales objetivos de la neurociencia.

Las técnicas de mapeo cerebral modernas (dMRI, fMRI, EEG, MEG), permiten obtener un mapa de las conexiones anatómicas o funcionales del cerebro. Para caracterizar estos mapeos, el análisis de redes complejas es el enfoque más utilizado. En este capítulo se presentará primero una breve explicación de la anatomía cerebral, luego se introducirá el concepto de conectoma (mapeo de conexiones cerebrales) y como se obtienen conectomas estructurales usando imágenes por tensor de difusión. Se dará también una breve explicación sobre redes y algunos resultados del análisis de las mismas en el campo de la neurociencia y por último se resumirán los objetivos del proyecto integrador.

\section{Cerebro humano}
\subsection{Neuronas}
% https://www.verywellmind.com/structure-of-a-neuron-2794896
% https://en.wikipedia.org/wiki/Neuron#Anatomy_and_histology
Las neuronas son células capaces de recibir, procesar y transmitir información a través de señales químicas y eléctricas mediante conexiones llamadas sinapsis. Son el principal componente del sistema nervioso en la mayoría de los animales y particularmente en los humanos. Anatómicamente, están formadas por el soma o cuerpo celular, las dendritas y el axón. El axón es la parte de la neurona encargada de transmitir la información hacia otras células, mientras que las dendritas y el soma se encargan de procesarla. En la figura \ref{fig:neurona_anatomia} se muestra un esquema de la estructura anatómica de una neurona.

\begin{figure}[htbp]
    \centering
    \includegraphics[width=.5\linewidth]{neurona_anatomia.png}
    \caption{Estructura anatómica de una neurona}
    \label{fig:neurona_anatomia}
\end{figure}

\subsection{Anatomía del cerebro humano}
% https://en.wikipedia.org/wiki/Nerve_tract
% https://en.wikipedia.org/wiki/Nucleus_(neuroanatomy)
% https://en.wikipedia.org/wiki/Human_brain
El cerebro humano esta formado por una capa exterior llamada corteza cerebral, compuesta principalmente por materia gris, y por un centro formado por materia blanca. La materia gris es un componente formado mayormente por los soma, dendritas y sinapsis de las neuronas, mientras que la materia blanca esta formada principalmente por axones. Conociendo las funciones de los axones y el soma, dendritas y sinapsis, se puede pensar, de forma simplificada, al cerebro como núcleos de neuronas (la materia gris de la corteza) comunicadas entre si por un tramado de fibras nerviosas (la materia blanca del núcleo). En la figura \ref{fig:cerebro} se muestra una representación de la estructura cerebral humana formada por materia gris y materia blanca.

\begin{figure}[htbp]
    \centering
    \includegraphics[width=.8\linewidth]{cerebro_materia_gris_blanca.jpg}
    \caption{Representación de la estructura cerebral humana}
    \label{fig:cerebro}
\end{figure}

\section{Conectomas}
% Introduction to Diffusion Tensor Imaging
% Building connectomes using diffusion MRI: why, how and but - Stamatios N. Sotiropoulos1,2 | Andrew Zalesky
Se llama conectoma al mapeo de conexiones neuronales entre regiones del cerebro humano. Las conexiones pueden ser:
\begin{itemize}
    \item Anatómicas o estructurales: representan las conexiones físicas (de materia blanca) entre regiones del cerebro.
    \item Funcionales: describen dependencias estadísticas entre las actividades neuronales entre regiones del cerebro.
\end{itemize}
Para obtener un conectoma estructural es necesario definir las regiones de interés del cerebro y obtener una medida de las fibras axonales. Las regiones de interés del cerebro se definen de acuerdo a un mapeos estándares llamados atlas cerebrales y las fibras axonales se pueden medir con imágenes por tensor de difusión y una tractografía.

\subsection{Imágenes por tensor de difusión}
Las imágenes por tensor de difusión permiten medir la difusión del agua en diferentes partes del cerebro, correspondientes a un píxel o vóxel en la imagen, al observar la variación de un campo magnético aplicado. Debido a que el agua tiende a difundir en la dirección de los tractos axonales del cerebro, conocer el tensor de difusión en una zona dada, nos brinda información sobre la dirección en que están orientados mayoritariamente los axones en esa zona. Ya que la resolución de la imagen no es del orden del tamaño de los axones, el tensor de difusión medido en un lugar del espacio es un promedio de la contribución de varias fibras axonales presentes en ese lugar.

\subsection{Tractografía}
La tractografía es la reconstrucción de tractos de fibras axonales a partir de las mediciones del tensor de difusión del agua. Hay tres pasos importantes a seguir para realizar una tractografía:
\begin{itemize}
    \item Primero hay que definir la estimación de la orientación axonal en un píxel o vóxel, para esto se puede usar el del tensor de difusión completo o simplemente la dirección principal del tensor de difusión. El último enfoque es mas directo pero inevitablemente se pierde información, sobre todo sobre la fiabilidad de esta dirección principal. Cuanto mas anisotrópico sea el tensor de difusión, mas confiable es la dirección principal del mismo como estimación de la orientación axonal. En el caso extremo en el que el tensor de difusión es completamente isotrópico (misma difusión en todas las direcciones), no se puede confiar en la dirección obtenida, ya que sería simplemente producto del ruido. En la figura \ref{fig:diffusion_estimation} se muestra el contraste entre las dos opciones para estimar la orientación axonal.

    \item El segundo paso es la propagación de los tractos basados en el campo de orientación axonal elegido, para esto hay varios métodos basados en interpolación o simulaciones.

    \item El tercer y último paso consiste en definir el final de un tracto, para esto los dos criterios mas usados son el de baja anisotropía y el de curva cerrada. En el criterio de baja anisotropía se asume que no hay una población coherente de fibras y en el de curva cerrada que la resolución de la imagen no es suficiente para detectar una curva de ese tipo. En la figura \ref{fig:tractografia_criterios} se muestra la forma en que termina la propagación de tractos de acuerdo a los dos criterios mencionados.

\end{itemize}


\begin{figure}[h]
    \centering
    \begin{subfigure}{\linewidth}
        \includegraphics[width=\linewidth]{diffusion_estimation.JPG}
        \caption{Representación de los campos de estimación de orientación axonal tensorial (A) y vectorial (B). El campo vectorial esta compuesto por las direcciones principales del campo tensorial en cada píxel o vóxel. En el caso en el que el tensor sea completamente isotrópico (píxel marcado con asterisco), el vector asociado no es confiable.}
        \label{fig:diffusion_estimation}
    \end{subfigure}
    \begin{subfigure}{\linewidth}
        \includegraphics[width=\linewidth]{tract_propagation.JPG}
        \caption{Reconstrucción de tractos a partir del campo de orientación axonal obtenido usando los criterios de baja anisotropía (A) y de curva cerrada (B). Los píxeles mas oscuros se corresponden con una anisotropía baja.}
        \label{fig:tract_propagation}
    \end{subfigure}
    \caption{Representación de los diferentes criterios posibles a la hora de realizar una tractografía.\cite{DTI_INTRODUCTION}}
    \label{fig:tractografia_criterios}
\end{figure}

\subsection{Regiones de Interés}
Definir la forma en la que se va a dividir el cerebro en regiones de interés discretas y espacialmente contiguas no es una tarea simple. En general, se puede dividir el cerebro en diferentes partes de acuerdo a la ubicación, anatomía y funciones con las que están relacionadas. La división obtenida depende también de la escala que se utilice, en una escala mas pequeña se pueden subdividir las regiones de interés presentes en una escala mas grande. Para obtener la división en una imagen dada se usan los atlas cerebrales, que permiten mapear múltiples vóxels de la imagen del cerebro a diferentes regiones de interés de acuerdo a una división modelo basándose en la posición, forma, tamaño y estructura anatómica.

\subsection{Construyendo un conectoma}
Una vez hecha la tractografía y la parcelación queda reconstruir las conexiones entre las diferentes regiones de interés. Lo mas común para este caso es asignarle a la conexión entre dos regiones un numero que dependa de la cantidad de tractos reconstruidos que las unan. También es común normalizar por el tamaño de cada región y por la fiabilidad de los tractos reconstruidos, de acuerdo a la anisotropía del tensor de difusión en casa vóxel que los conforman.

\section{Análisis de redes cerebrales}
Las redes o grafos son una forma de modelar matemáticamente diversos sistemas complejos formados por elementos fundamentales conectados entre si. Una red esta formada por un conjunto de nodos y conexiones o aristas entre pares de nodos. En el caso de redes cerebrales, los nodos representan a las regiones de interés de un conectoma y las aristas a las conexiones estructurales o funcionales que fueron obtenidas entre esas regiones con los métodos explicados anteriormente. El análisis de redes cerebrales ha llevado a diversos resultados importantes, Goñi et al. pudo predecir con bastante precisión la conectividad funcional utilizando medidas de comunicación sobre conectomas estructurales \cite{GoniFunctionalPrediction} y Crofts et al. encontró diferencias sustanciales en las redes estructurales de pacientes que sufrieron un accidente cerebrovascular \cite{CroftsStroke}. Estos resultados demuestran que el análisis de redes cerebrales puede ser una herramienta de suma importancia para caracterizar patologías y entender mejor como funciona el cerebro.

\section{Objetivos y materiales}
El objetivo principal del presente proyecto es el de aplicar medidas de comunicación de redes complejas sobre conectomas estructurales buscando diferencias entre una población de pacientes con migraña crónica y una población de controles sanos. Para esto, se utilizan técnicas de análisis de redes complejas.

\subsection{Materiales y herramientas}
El material con el que se trabajó en el presente proyecto consiste en matrices de conexión estructural correspondientes a 72 pacientes con migraña episódica, 69 pacientes con migraña crónica y 62 sujetos de control.

El análisis de las redes cerebrales se realizó en el lenguaje de programación \textbf{Python 3} utilizando las librerías \textbf{Networkx} y \textbf{Numpy} principalmente.